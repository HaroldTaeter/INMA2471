\section*{Question 2}
Il s'agit ici de se donner une idée de la robustesse de la solution du modèle linéaire. On va donc supposer ici que les facteurs $x_i$ sont entachés d'une erreur relative $\xi_i$. On a donc le véritable diagramme exprimé comme $\hat{D}(\theta) = sum_{i=1}^{N}{(1+\xi_i)x_i d_i(\theta)}$. On va supposer que les $\xi_i$ sont ici distribués aléatoirement de manière uniforme sur un intervalle $[-\tau,\tau]$, indépendamment les uns des autres. La démarche consiste donc à résoudre le modèle de la première question et d'utiliser ensuite les $x_i$ calculés pour se donner une idée de la fonction $\hat{D}$.\\
Il est demandé d'examiner les diagrammes d'antenne pour $\tau = 0.001$ et pour $\tau = 0.01$. On se rend très vite compte qu'en plottant $\hat{D}(\theta)$, on obtient des résultats sensiblement différents de ceux de la question précédente. \red{Ces résultats sont présentés sur les figures...}. Pour les erreurs relatives à priori pas trop élevées, on se retrouve avec des résultats complètement abérrants. La solution du modèle linéaire est donc très peu robuste.
\red{+ tenter de l'expliquer intuitivement}