\documentclass[11pt,a4paper]{article}

\usepackage[T1]{fontenc}
\usepackage[utf8]{inputenc}
\usepackage[french]{babel}
\usepackage{lmodern}
%\usepackage{circuitikz}
\usepackage{color}
\usepackage{wrapfig}
\usepackage{subfigure}
\usepackage{fullpage}
\usepackage[squaren]{SIunits}
\usepackage{graphicx}
%\usepackage[pdftex]{graphicx}
\usepackage{epstopdf}
\usepackage{epsfig}
\usepackage{hyperref}
\usepackage{eurosym}
%\usepackage{chemist}
\usepackage{amsmath}
\usepackage{amssymb}
\usepackage{mathrsfs}
\usepackage{dsfont}% use $\mathds{1}$
\newcommand{\C}{\mathbb{C}}
\newcommand{\N}{\mathbb{N}}
\newcommand{\Z}{\mathbb{Z}}
\newcommand{\R}{\mathbb{R}}
\newcommand{\red}{\textcolor{red}}
\newcommand{\dis}{\displaystyle}
\newcommand{\dr}{\partial}
\newcommand{\txt}{\text}

%\newtheorem{theoreme}			     {Théorème}	[chapter]
%\newtheorem{proposition}[theoreme]	 {Proposition}	
%\newtheorem{corollaire}	  [theoreme]	 {Corollaire}	
%\newtheorem{lemme}	      [theoreme]  {Lemme}		
%\newtheorem{definition}	         {Définition}[chapter]
%\theoremstyle{definition}
%\newtheorem{exemple}			     {Exemple}	[chapter]
%\newtheorem{contreexemple}[exemple]{Contre-exemple}
%\newtheorem{probleme}	             {Probl\`eme}[chapter]

\usepackage{listings}
\usepackage{textcomp}
\definecolor{listinggray}{gray}{0.9}
\definecolor{lbcolor}{rgb}{0.9,0.9,0.9}
\lstset{
	backgroundcolor=\color{lbcolor},
	tabsize=4,
	rulecolor=,
	language=matlab,
        basicstyle=\scriptsize,
        upquote=true,
        aboveskip={1.5\baselineskip},
        columns=fixed,
        showstringspaces=false,
        extendedchars=true,
        breaklines=true,
        prebreak = \raisebox{0ex}[0ex][0ex]{\ensuremath{\hookleftarrow}},
        frame=single,
        showtabs=false,
        showspaces=false,
        showstringspaces=false,
        identifierstyle=\ttfamily,
        keywordstyle=\color[rgb]{0,0,1},
        commentstyle=\color[rgb]{0.133,0.545,0.133},
        stringstyle=\color[rgb]{0.627,0.126,0.941},
}

\DeclareMathOperator{\e}{e}

\title{Titre}
\author{Florentin Goyens}
\date{\today}

\begin{document}

\begin{center}
\hrule
\begin{tabular}{c}
\\[0.005cm]
\Large{Modèles et méthodes d'optimisation - Devoir 2}\\[0.3cm]
\textsc{Goyens} Florentin \\[0.2cm] \textsc{Taeter} Harold \\[0.2cm]
$\text{17}^{\text{th}}$ November 2014\\[0.2cm]
\end{tabular}
\hrule
\end{center}




\section*{Question 1}
 
 Dans la première partie du travail, nous implémentons la version standard du modèle, qui ne tient pas encore compte de la robustesse. Le code \textsc{Ampl} est joint au rapport. 
 
 \subsection*{Approximation du problème posé}
L'approximation s'effectue en deux endroits. Dans un premier temps, les valeurs $d_{i}(\theta)$ sont calculée numériquement car on ne dispose pas d'une formule exacte pour l'intégrale. Cela constitue une première approximation mineure. Nous avons utilisé \textsc{Matlab} et sa fonction \texttt{quad}. \red{On exprime cela comme une fonction de Bessel ??}.

Une seconde erreur, plus remarquable, vient de la \underline{relaxation du problème}. 
Les contraintes $|D(\theta)-1| \leq \varepsilon\text{ } \forall \theta\in [\theta_{\mathcal{P}} 90\degree]$ et $|D(\theta)| \leq \varepsilon\text{ } \forall \theta\in [0 \theta_{S}]$ cachent en fait une infinité de contraintes puisqu'il faut les appliquer pour tous les angles dans les intervals considérés.
On relâche le problème en choisissant un nombre fini de contraintes, on applique les conditions sur un sous ensemble discret des intervals $[\theta_{\mathcal{P}} 90\degree]$ et $[0 \theta_{S}]$. En pratique, nous avons sélectionné toutes les valeurs entières des angles. \red{TODO: vérifier de la manière la plus précise possible que cela est une assez bonne approximation}.


\subsection*{Solution du cas de base}


\subsection*{Variation des paramètres $\theta_{P}$ et $\theta_{S}$}
On s'attend logiquement a une moins bonne approximation lorsque les deux valeurs se rapprochent. En effet, cela augmente le nombre de contraintes et donc réduit l'espace admissible pour minimiser $\varepsilon$. Plus graphiquement, on réduit la largeur de la bande de transition de l'échelon du diagramme et cela rend plus difficile le passage de la valeur $0$ à la valeur $1$. La figure~\ref{chgmtparam} montre ce phénomène. Il est visible que les oscillations sont plus grandes lorsque l'on rapproche les paramètres. Pour le quantifier, on dresse la table des valeurs optimales de la bande $\varepsilon$.



\begin{tabular}{|c|c|c|}
\hline 
$\theta_{S}$ $[\degree]$ & $\theta_{P}$ $[\degree]$ & $\varepsilon^{*}$ \\ 
\hline 
40 & 50 & 0.01716135157 \\ 
\hline 
42 & 48 & 0.0736852368 \\ 
\hline 
43 & 47 &  \\ 
\hline 
44 & 45 & 0.1180329256 \\
\hline 
40 & 45 & \\
\hline 
40 & 42 &  \\
\hline 
48 & 50 & \\
\hline 
\end{tabular} 



%\begin{tabular}{|c|c|c|}
%\hline 
%$\theta_{S}$ $[\degree]$ & $\theta_{P}$ $[\degree]$ & $\varepsilon^{*}$ \\ 
%\hline 
%40 & 50 & 0.02512647508 \\ 
%\hline 
%42 & 48 & 0.05265152209 \\ 
%\hline 
%43 & 47 & 0.1032455449 \\ 
%\hline 
%44 & 45 & 0.1634615725 \\
%\hline 
%40 & 45 & 0.07309577599 \\
%\hline 
%40 & 42 & 0.1636404621 \\
%\hline 
%48 & 50 &  0.131448618 \\
%\hline 
%\end{tabular} 




\subsection*{Calcul}
\red{todo: parler du temps calcul (ampl/matlab) et des solveurs}.











\section*{Question 2}
Il s'agit ici de se donner une idée de la robustesse de la solution du modèle linéaire. On va donc supposer ici que les facteurs $x_i$ sont entachés d'une erreur relative $\xi_i$. On a donc le véritable diagramme exprimé comme $\hat{D}(\theta) = sum_{i=1}^{N}{(1+\xi_i)x_i d_i(\theta)}$. On va supposer que les $\xi_i$ sont ici distribués aléatoirement de manière uniforme sur un intervalle $[-\tau,\tau]$, indépendamment les uns des autres. La démarche consiste donc à résoudre le modèle de la première question et d'utiliser ensuite les $x_i$ calculés pour se donner une idée de la fonction $\hat{D}$.\\
Il est demandé d'examiner les diagrammes d'antenne pour $\tau = 0.001$ et pour $\tau = 0.01$. On se rend très vite compte qu'en plottant $\hat{D}(\theta)$, on obtient des résultats sensiblement différents de ceux de la question précédente. \red{Ces résultats sont présentés sur les figures...}. Pour les erreurs relatives à priori pas trop élevées, on se retrouve avec des résultats complètement abérrants. La solution du modèle linéaire est donc très peu robuste.
\red{+ tenter de l'expliquer intuitivement}

\section*{Question 3}

\section*{Question 4}

Le principal problème de la formulation développée précédemment est qu'elle est trop conservatrice. En effet, on supposait qu'on se trouvait toujours dans le pire cas. Cependant, il est très peu probable que cela se réalise. On va donc plutôt considérer la condition suivante : $$\sum_{i=1}^{2N}{\xi_i^2} \leq \gamma^2$$ où $\gamma > 0$ est un paramètre ajustable.\\
On va ici chercher un $\gamma$ qui permettra de satisfaire cette contrainte avec une probabilité de plus de $99.99\%$.

\end{document}