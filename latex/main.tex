\documentclass[11pt,a4paper]{article}

\usepackage[T1]{fontenc}
\usepackage[utf8]{inputenc}
\usepackage[french]{babel}
\usepackage{lmodern}
%\usepackage{circuitikz}
\usepackage{color}
\usepackage{wrapfig}
\usepackage{subfigure}
\usepackage{fullpage}
\usepackage[squaren]{SIunits}
\usepackage{graphicx}
%\usepackage[pdftex]{graphicx}
\usepackage{epstopdf}
\usepackage{epsfig}
\usepackage{hyperref}
\usepackage{eurosym}
%\usepackage{chemist}
\usepackage{amsmath}
\usepackage{amssymb}
\usepackage{mathrsfs}
\usepackage{dsfont}% use $\mathds{1}$
\newcommand{\C}{\mathbb{C}}
\newcommand{\N}{\mathbb{N}}
\newcommand{\Z}{\mathbb{Z}}
\newcommand{\R}{\mathbb{R}}
\newcommand{\red}{\textcolor{red}}
\newcommand{\dis}{\displaystyle}
\newcommand{\dr}{\partial}
\newcommand{\txt}{\text}

%\newtheorem{theoreme}			     {Théorème}	[chapter]
%\newtheorem{proposition}[theoreme]	 {Proposition}	
%\newtheorem{corollaire}	  [theoreme]	 {Corollaire}	
%\newtheorem{lemme}	      [theoreme]  {Lemme}		
%\newtheorem{definition}	         {Définition}[chapter]
%\theoremstyle{definition}
%\newtheorem{exemple}			     {Exemple}	[chapter]
%\newtheorem{contreexemple}[exemple]{Contre-exemple}
%\newtheorem{probleme}	             {Probl\`eme}[chapter]

\usepackage{listings}
\usepackage{textcomp}
\definecolor{listinggray}{gray}{0.9}
\definecolor{lbcolor}{rgb}{0.9,0.9,0.9}
\lstset{
	backgroundcolor=\color{lbcolor},
	tabsize=4,
	rulecolor=,
	language=matlab,
        basicstyle=\scriptsize,
        upquote=true,
        aboveskip={1.5\baselineskip},
        columns=fixed,
        showstringspaces=false,
        extendedchars=true,
        breaklines=true,
        prebreak = \raisebox{0ex}[0ex][0ex]{\ensuremath{\hookleftarrow}},
        frame=single,
        showtabs=false,
        showspaces=false,
        showstringspaces=false,
        identifierstyle=\ttfamily,
        keywordstyle=\color[rgb]{0,0,1},
        commentstyle=\color[rgb]{0.133,0.545,0.133},
        stringstyle=\color[rgb]{0.627,0.126,0.941},
}

\DeclareMathOperator{\e}{e}

\title{Titre}
\author{Florentin Goyens}
\date{\today}

\begin{document}

\begin{center}
\hrule
\begin{tabular}{c}
\\[0.005cm]
\Large{Modèles et méthodes d'optimisation - Devoir 2}\\[0.3cm]
\textsc{Goyens} Florentin \\[0.2cm] \textsc{Taeter} Harold \\[0.2cm]
$\text{17}^{\text{th}}$ November 2014\\[0.2cm]
\end{tabular}
\hrule
\end{center}




\section*{Question 1}
 
 Dans la première partie du travail, nous implémentons la version standard du modèle, qui ne tient pas encore compte de la robustesse.
















\section*{Question 2}
Il s'agit ici de se donner une idée de la robustesse de la solution du modèle linéaire. On va donc supposer ici que les facteurs $x_i$ sont entachés d'une erreur relative $\xi_i$. On a donc le véritable diagramme exprimé comme $\hat{D}(\theta) = sum_{i=1}^{N}{(1+\xi_i)x_i d_i(\theta)}$. On va supposer que les $\xi_i$ sont ici distribués aléatoirement de manière uniforme sur un intervalle $[-\tau,\tau]$, indépendamment les uns des autres. La démarche consiste donc à résoudre le modèle de la première question et d'utiliser ensuite les $x_i$ calculés pour se donner une idée de la fonction $\hat{D}$.\\
Il est demandé d'examiner les diagrammes d'antenne pour $\tau = 0.001$ et pour $\tau = 0.01$. On se rend très vite compte qu'en plottant $\hat{D}(\theta)$, on obtient des résultats sensiblement différents de ceux de la question précédente. \red{Ces résultats sont présentés sur les figures...}. Pour les erreurs relatives à priori pas trop élevées, on se retrouve avec des résultats complètement abérrants. La solution du modèle linéaire est donc très peu robuste.
\red{+ tenter de l'expliquer intuitivement}

\section*{Question 3}
A la lumière des résultats présentés précédemment, il est clair qu'on va chercher à établir une version plus robuste du problème. On se propose ici de présenter une première formulation.\\
Gardant en tête que les valeurs $x_i$ vont être entachées d'erreurs, on a le problème d'optimisation suivant :
\begin{align*}
\min \epsilon &\\
\text{s.c.} & \quad \sum_{i=1}^{N}{(1+\xi_i)x_i d_i(\theta)}-1 \leq \epsilon, \quad \theta \in [\theta_P,90\degree],\\
 & \quad \sum_{i=1}^{N}{(1+\xi_i)x_i d_i(\theta)}-1 \geq -\epsilon, \quad \theta \in [\theta_P,90\degree],\\
 & \quad \sum_{i=1}^{N}{(1+\xi_i)x_i d_i(\theta)} \leq \epsilon, \quad \theta \in [0\degree,\theta_S],\\
 & \quad \sum_{i=1}^{N}{(1+\xi_i)x_i d_i(\theta)} \geq -\epsilon, \quad \theta \in [0\degree,\theta_S],
\end{align*}
dans lequel les données $\xi_i$ sont des réalisations d'une variable aléatoire.\\
Afin de nous débarasser de ces valeurs aléatoires, nous allons reformuler nos contraintes en nous plaçant dans le pire cas.\\
Occupons-nous tout d'abord de la première contrainte : $$\sum_{i=1}^{N}{(1+\xi_i)x_i d_i(\theta)}-1 \leq \epsilon, \quad \theta \in [\theta_P,90\degree].$$
Celle-ci peut se réécrire $$\sum_{i=1}^{N}{\xi_i x_i d_i(\theta)} \leq 1+\epsilon-\sum_{i=1}^{N}{x_i d_i(\theta)}, \quad \theta \in [\theta_P,90\degree].$$
Le terme $\sum_{i=1}^{N}{\xi_i x_i d_i(\theta)}$ est responsable de la différence entre le véritable diagramme et celui calculé sur base du problème linéaire. Le pire cas consiste donc à choisir les $\xi_i$ qui maximisent ce terme. On va minimiser l'erreur qu'on peut obtenir dans le pire des cas. Afin de se placer dans cette situation, on écrit la contrainte de la manière suivante : $$\Big(\max_{\xi_i} \sum_{i=1}^{N}{\xi_i x_i d_i(\theta)}\Big) \leq 1+\epsilon-\sum_{i=1}^{N}{x_i d_i(\theta)}, \quad \theta \in [\theta_P,90\degree],$$ tout en gardant à l'esprit que $$-\tau \leq \xi_i \leq \tau, \quad \forall i.$$
Cette contrainte peut donc être vue comme un problème d'optimisation et donc transformée en utilisant le principe de dualité forte. Après application de ce principe\footnote{\red{Je ne suis pas sur que tout ce qui était écrit sur la feuille de flo était ok. Après dualité forte, on a $\min d_i^T x_i \leq c_i$. Ca devrait pas être un $\geq$ ? Et on maximisait sur $a_i$, maintenant on minimise sur $x$, un peu bizarre. A la fin, on met $d_i^T x_i = c_i$, est-ce que c'est bien un $=$ ? (et quel interet de mettre un indice à $c$ ?)}}, la contrainte se réécrit : $$\Big(\min_{y_i\red{?}} \tau \sum_{j=1}^{2N}{y_i}\Big) \leq 1+\epsilon-\sum_{i=1}^{N}{x_i d_i(\theta)}, \quad \theta \in [\theta_P,90\degree],$$ 
$$\begin{pmatrix}
y_1-y_{N+1} & y_2-y_{N+2} & \hdots & y_N-y_{2N}
\end{pmatrix}
=
\begin{pmatrix}
x_1 d_1(\theta) & x_2 d_2(\theta) & \hdots & x_N d_N(\theta)
\end{pmatrix},
\quad \theta \in [\theta_P,90\degree],$$
$$y_i \geq 0, \quad \forall i.$$
En d'autres termes, il existe $y \in \mathbb{R}_{+}^{2N}$ tel que
\begin{align*}
\tau \sum_{j=1}^{2N}{y_i} & = 1+\epsilon-\sum_{i=1}^{N}{x_i d_i(\theta)}, \quad \theta \in [\theta_P,90\degree],\\
\begin{pmatrix}
y_1-y_{N+1} & y_2-y_{N+2} & \hdots & y_N-y_{2N}
\end{pmatrix}
& =
\begin{pmatrix}
x_1 d_1(\theta) & x_2 d_2(\theta) & \hdots & x_N d_N(\theta)
\end{pmatrix},
\quad \theta \in [\theta_P,90\degree].
\end{align*}
On ajoute donc les variables $y_i$ dans le modèle et on peut alors reformuler la contrainte de départ de la manière mentionnée ci-dessus. En effectuant la démarche ci-dessus pour les quatre sortes de contraintes du modèle de départ, on se retrouver avec un modèle plus robuste puis qu'il se base toujours sur le pire cas.


\section*{Question 4}

Le principal problème de la formulation développée précédemment est qu'elle est trop conservatrice. En effet, on supposait qu'on se trouvait toujours dans le pire cas. Cependant, il est très peu probable que cela se réalise. On va donc plutôt considérer la condition suivante : $$\sum_{i=1}^{2N}{\xi_i^2} \leq \gamma^2$$ où $\gamma > 0$ est un paramètre ajustable.\\
On va ici chercher un $\gamma$ qui permettra de satisfaire cette contrainte avec une probabilité de plus de $99.99\%$.

\end{document}